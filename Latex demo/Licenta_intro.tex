\documentclass{article}

\makeatletter
\def\@seccntformat#1{%
  \expandafter\ifx\csname c@#1\endcsname\c@section\else
  \csname the#1\endcsname\quad
  \fi}
\makeatother

\title{\bf Lucrare de licen\c{t}\u{a}}
\begin{document}
  \pagenumbering{gobble}
  \maketitle
  \newpage
  \pagenumbering{arabic}

  \section {Introducere}

\^{I}n conformitate cu Regulamentul de organizare \c{s}i desf\u{a}\c{s}urare a examenului de
absolvire/licenţ\u{a}/diplom\u{a} la Universitatea Alexandru Ioan Cuza Ia\c{s}i, studiile \^{i}n
\^{i}nv\u{a}\c{t}\u{a}mântul universitar de licenţ\u{a} se \^{i}ncheie cu examen licenţ\u{a}. Examenul de licenţ\u{a}
const\u{a} din dou\u{a} probe: 

\begin{itemize}
  \item evaluarea cuno\c{s}tin\c{t}elor fundamentale \c{s}i de specialitate; 
  \item evaluarea prezent\u{a}rii \c{s}i sus\c{t}inerii lucr\u{a}rii de licen\c{t}\u{a}. 
\end{itemize}

Fiecare student al Facult\u{a}ţii de Informatic\u{a} are obligaţia de a-şi alege tema lucr\u{a}rii de
licenţ\u{a} şi cadrul didactic coordonator, respectând, pe parcursul elabor\u{a}rii şi prezent\u{a}rii
lucr\u{a}rii, cerinţele impuse de acesta.

Prezentul Ghid pentru examenul Ghid pentru examenul examenul de licenţ\u{a} de licenţ\u{a} (numit \^{i}n continuare Ghid) este un
document adoptat de c\u{a}tre Consiliul Facult\u{a}ţii de Informatic\u{a}, având urm\u{a}toarele
obiective:

\begin{itemize}
	\item facilitarea redact\u{a}rii corecte de c\u{a}tre student/absolvent a lucr\u{a}rii de licenţ\u{a};
	\item creşterea nivelului calitativ al lucr\u{a}rilor de licenţ\u{a};
	\item evaluarea unitar\u{a} a absolvenţilor care \^{i}şi susţin examenul de licenţ\u{a}. 
\end{itemize}

Prezentul Ghid va fi utilizat ca instrument de lucru al Facult\u{a}ţii de Informatic\u{a} \^{i}ncepând
cu sesiunea de licenţ\u{a} iulie 2009 iulie 2009.

Ghidul, \^{i}mpreun\u{a} cu şabloanele utilizate pentru redactarea lucr\u{a}rii de licenţ\u{a}, va fi
publicat pe situl Facult\u{a}ţii de Informatic\u{a} (http://www.infoiasi.ro) şi va fi diseminat de
c\u{a}tre cadrele didactice coordonatoare tuturor studenţilor/absolvenţilor pe care acestea \^{i}i
\^{i}ndrum\u{a}. 


\section {Evaluarea cunoştinţelor fundamentale şi de specialitate}

Evaluarea cunoştinţelor fundamentale şi de specialitate se realizeaz\u{a} imediat dup\u{a}
prezentarea lucr\u{a}rii de licenţ\u{a} pe baza unui set de \^{i}ntreb\u{a}ri menite s\u{a} testeze nivelul de
cunoaştere şi st\u{a}pânire a conceptelor fundamentale utilizate \^{i}n lucrare sau care au
strâns\u{a} leg\u{a}tur\u{a} cu subiectele tratate \^{i}n lucrare. Fiecare membru al comisiei de
examinare va pune o \^{i}ntrebare. Pe parcursul formul\u{a}rii r\u{a}spunsului de c\u{a}tre candidat,
membrii comisiei pot solicita l\u{a}muriri suplimentare. Fiecare membru al comisiei
evalueaz\u{a} fiecare r\u{a}spuns şi va acorda o not\u{a} ce va constitui media acestor evalu\u{a}ri.
Evaluarea r\u{a}spunsului va urm\u{a}ri \^{i}n mod preponderent nivelul de \^{i}nţelegere a
conceptelor, capacitatea de dezvoltare a unui discurs coerent şi riguros, corelarea cu
alte domenii şi nu redarea cu exactitate a unor definiţii sau demonstraţii predate la
disciplinele respective. 

\section {Lucrarea de licenţ\u{a}}

 \begin{enumerate}
   \item Copert\u{a} – informaţiile care trebuie s\u{a} apar\u{a} pe coperta lucr\u{a}rii de licenţ\u{a}sunt prezentate \^{i}n Anexa 2;
   \item Pagina de titlu – informaţiile care trebuie s\u{a} apar\u{a} \^{i}n pagina de titlu a lucr\u{a}rii de licenţ\u{a} sunt prezentate \^{i}n Anexa 3. 
   \item Declaraţii standard – lucrarea de licenţ\u{a} va conţine dou\u{a} declaraţii:
   \begin{enumerate}
     \item o declaraţie pe propria r\u{a}spundere a absolventului, datat\u{a} şi
	semnat\u{a} \^{i}n original, din care s\u{a} rezulte c\u{a} lucrarea \^{i}i aparţine, nu a
	mai fost niciodat\u{a} prezentat\u{a} şi nu este plagiat\u{a}. Conţinutul
	declaraţiei este prezentat \^{i}n Anexa 4
     \item o declaraţie pe propria r\u{a}spundere a absolventului, datat\u{a} şi
	semnat\u{a} \^{i}n original, din care s\u{a} rezulte c\u{a} este de acord ca
	lucrarea şi codul surs\u{a} a programelor s\u{a} poat\u{a} fi utilizate \^{i}n cadrul
	Facult\u{a}ţii de Informatic\u{a} \^{i}n scopuri educaţionale. \^{I}n cazul \^{i}n care 
	exist\u{a} un acord \^{i}ntre absolvent şi Facultatea de Informatic\u{a} privind
	dreptului de autor asupra codului surs\u{a}, atunci acel acord va fi
	parte integrant\u{a} a prezentei declaraţii. Conţinutul declaraţiei este
	prezentat \^{i}n Anexa 5. 
	Introducere – aceasta va conţine motivaţia alegerii temei, gradul de
	noutate a temei, obiectivele generale ale lucr\u{a}rii, metodologia folosit\u{a},
	descrierea sumar\u{a} a soluţiei, structura lucr\u{a}rii (titlul capitolelor şi leg\u{a}tura
	dintre ele). Introducerea nu se numeroteaz\u{a} ca şi capitol.
      \item Contribuţii – aceasta va avea cel mult o pagin\u{a} şi va descrie schematic
	principalele contribuţii ale absolventului \^{i}n realizarea lucr\u{a}rii.
      \item Capitole – lucrarea de licenţ\u{a} va conţine capitole numerotate cresc\u{a}tor,
	fiecare putând s\u{a} aib\u{a}, \^{i}n partea final\u{a}, o secţiune de concluzii, care s\u{a}
	sintetizeze informaţiile şi/sau rezultatele prezentate \^{i}n cadrul acelui
	capitol.
      \item Concluziile lucr\u{a}rii – \^{i}n aceast\u{a} parte a lucr\u{a}rii de licenţ\u{a} se reg\u{a}sesc cele
	mai importante concluzii din lucrare, opinia personal\u{a} privind rezultatele
	obţinute \^{i}n lucrare, precum şi potenţiale direcţii viitoare de cercetare
	legate de tema abordat\u{a}. Concluziile lucr\u{a}rii nu se numeroteaz\u{a} ca şi
	capitol.
      \item Bibliografie – acesta este ultima parte a lucr\u{a}rii şi va conţine lista tuturor
	surselor de informaţie utilizate de c\u{a}tre absolvent pentru redactarea
	lucr\u{a}rii de licenţ\u{a}. Bibliografia nu se va numerota ca şi capitol al lucr\u{a}rii.
      \item Anexe (dac\u{a} este cazul) – acestea apar dup\u{a} bibliografie, care nu se
	numeroteaz\u{a} ca şi capitol. Fiecare anex\u{a} se va menţiona cel puţin o dat\u{a}
	\^{i}n textul lucr\u{a}rii. Anexele se numeroteaz\u{a} cresc\u{a}tor (Anexa 1, Anexa 2,
	etc.). 
   \end{enumerate}
 \end{enumerate}

\end{document}